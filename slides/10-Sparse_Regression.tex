% !TeX root = slides.tex
\section{Sparse Regression}

\subsection{Recommended References}
\begin{frame}{Recommended References}
	\begin{vfilleditems}
		\item \textcite{gelman2020regression} - Chapter 12, Section 12.8: Models for regression coefficients
		\item Horseshoe Prior: \textcite{carvalho2009handling}
		\item Horseshoe+ Prior: \textcite{bhadra2015horseshoe}
        \item Regularized Horseshoe Prior: \textcite{piironen2017horseshoe}
        \item R2-D2 Prior: \textcite{zhang2022bayesian}
        \item Betancourt's Case study on Sparsity: \textcite{betancourtSparsityBlues2021}
	\end{vfilleditems}
\end{frame}

\subsection{Sparsity}
\begin{frame}{What is Sparsity?}
    Sparsity is a concept frequently encountered in statistics, signal processing,
    and machine learning, which refers to situations where the vast majority of
    elements in a dataset or a vector are zero or close to zero. 
\end{frame}

\begin{frame}{How to Handle Sparsity?}
    Almost all techniques deal with some sort of \textbf{variable selection},
    instead of altering data.

    This makes sense from a Bayesian perspective,
    as data is \textbf{information},
    and we don't want to throw information away.
\end{frame}

\subsection{Frequentist Approach}
\begin{frame}{Frequentist Approach}
    The frequentist approach deals with sparse regression by staying in the
    ``optimization'' context but adding \textbf{Lagrangian constraints}\footnote{
        this is called \textbf{LASSO} (least absolute shrinkage and selection operator)
        from \textcite{tibshirani1996regression,zou2005regularization}.
    }:

    $$
        \min_\beta \left\{ \sum _{i=1}^N ( y_i - \alpha - x_i^T \boldsymbol{\beta} )^2 \right\}
    $$

    suject to $\vert \boldsymbol{\beta} \vert_p \leq t$.

    \vfill

    Here $\vert \cdot \vert_p$ is the $p$-norm.
\end{frame}

\subsection{Variable Selection Techniques}
\begin{frame}{Variable Selection Techniques}
    \begin{vfilleditems}
        \item \textbf{discrete mixtures}: \textit{spike-and-slab} prior
        \item \textbf{shrinkage priors}: \textit{Laplace} prior
              and \textit{horseshoe} prior \parencite{carvalho2009handling}
    \end{vfilleditems}
\end{frame}

\subsection{Discrete Mixtures}
\subsubsection{Spike-and-Slab Prior}
\begin{frame}{Discrete Mixtures -- Spike-and-Slab Prior}
    \small
    Mixture of two distributions—one that is
    concentrated at zero (the ``spike'') and one with a much wider spread (the
    ``slab''). This prior indicates that we believe most coefficients in our model are
    likely to be zero (or close to zero), but we allow the possibility that some are
    not.

    \vfill 
    Here is the Gaussian case:

    $$
        \begin{aligned}
            \beta_i \mid \lambda_i, c &\sim \text{Normal} \left( 0, \sqrt{\lambda_i^2 c^2} \right) \\
            \lambda_i &\sim \text{Bernoulli} (p)
        \end{aligned}
    $$

    where:
    \begin{vfilleditems}
        \small
        \item $c$: \textit{slab} width
        \item $p$: prior inclusion probability;
            encodes the prior information about the sparsity of the
            coefficient vector $\boldsymbol{\beta}$
        \item $\lambda_i \in \{0, 1\}$: whether the coefficient $\beta_i$ is close
            to zero (comes from the ``spike'', $\lambda_i = 0$) or nonzero
            (comes from the ``slab'', $\lambda_i = 1$)
    \end{vfilleditems}
\end{frame}

\begin{frame}{Discrete Mixtures -- Spike-and-Slab Prior}
	\centering
	\begin{tikzpicture}
		\begin{axis}[every axis plot, line width=2pt,
				ylabel=PDF,
				domain=-4:4,samples=200,
				axis x line*=bottom, % no box around the plot, only x and y axis
				axis y line*=left, % the * suppresses the arrow tips
				enlarge x limits=true, % extend the axes a bit
			]

            \addplot [blue] {hs(1,0.05};
			\addlegendentry{spike: $c=1, \lambda=0$}
			\addplot [red] {hs(1,1)};
			\addlegendentry{slab: $c=1, \lambda=1$}
		\end{axis}
	\end{tikzpicture}
\end{frame}

\subsection{Shinkrage Priors}
\subsubsection{Laplace Prior}
\begin{frame}{Shinkrage Priors -- Laplace Prior}
    The Laplace distribution is a continuous probability distribution named after
    Pierre-Simon Laplace.
    It is also known as the double exponential distribution.

    It has parameters:
    \begin{vfilleditems}
        \item $\mu$: location parameter
        \item $b$: scale parameter
    \end{vfilleditems}

    The PDF is:

    $$
        \text{Laplace}(\mu, b) = \frac{1}{2b} e^{-\frac{| x - \mu |}{b}}
    $$

    \vfill

    It is a symmetrical exponential decay around $\mu$ with scale governed by $b$.
\end{frame}

\begin{frame}{Shinkrage Priors -- Laplace Prior}
	\centering
	\begin{tikzpicture}
		\begin{axis}[every axis plot, line width=2pt,
				ylabel=PDF,
				domain=-4:4,samples=200,
				axis x line*=bottom, % no box around the plot, only x and y axis
				axis y line*=left, % the * suppresses the arrow tips
				enlarge x limits=true, % extend the axes a bit
			]

			\addplot [blue] {laplace(1)};
			\addlegendentry{$\mu=0, b=1$}
		\end{axis}
	\end{tikzpicture}
\end{frame}

\subsubsection{Horseshoe Prior}
\begin{frame}{Shinkrage Priors -- Horseshoe Prior}
    The horseshoe prior\parencite{carvalho2009handling} assumes that each coefficient
    $\beta_i$ is conditionally independent with density
    $P_{\text{HS}}(\beta_i \mid \tau )$, where $P_{\text{HS}}$
    can be represented as a scale mixture of Gaussians:

    $$
        \begin{aligned}
            \beta_i \mid \lambda_i, \tau &\sim \text{Normal} \left( 0, \sqrt{\lambda_i^2 \tau^2} \right) \\
            \lambda_i &\sim \text{Cauchy}^+ (0, 1)
        \end{aligned}
    $$

    where:
    \begin{vfilleditems}
        \item $\tau$: \textit{global} shrinkage parameter
        \item $\lambda_i$: \textit{local} shrinkage parameter
        \item $\text{Cauchy}^+$ is the half-Cauchy distribution for the
            standard deviation $\lambda_i$
    \end{vfilleditems}

    \vfill

    \small
    Note that it is similar to the spike-and-slab, but the discrete mixture becomes
    a ``continuous'' mixture with the $\text{Cauchy}^+$.
\end{frame}

\begin{frame}{Shinkrage Priors -- Horseshoe}
	\centering
	\begin{tikzpicture}
		\begin{axis}[every axis plot, line width=2pt,
				ylabel=PDF,
				domain=-4:4,samples=200,
				axis x line*=bottom, % no box around the plot, only x and y axis
				axis y line*=left, % the * suppresses the arrow tips
				enlarge x limits=true, % extend the axes a bit
			]

            \addplot [blue] {hs(1,1};
			\addlegendentry{$\tau=1, \lambda=1$}
			\addplot [red] {hs(1,0.5)};
			\addlegendentry{$\tau=1, \lambda=0.5$}
		\end{axis}
	\end{tikzpicture}
\end{frame}

\subsubsection{Discrete Mixtures versus Shinkrage Priors}
\begin{frame}{Discrete Mixtures versus Shinkrage Priors}
    \textbf{Discrete mixtures} offer the correct representation of sparse problems
    \parencite{carvalho2009handling} by placing positive prior probability on
    $\beta_i = 0$ (regression coefficient),
    but pose several difficulties: mostly computational due to the
    \textbf{non-continuous nature}.

    \textbf{Shrinkage priors}, despite not having the best representation of sparsity,
    can be very attractive computationally: again due to the \textbf{continuous property}.
\end{frame}

\begin{frame}{Horseshoe versus Laplace}
    The advantages of the Horseshoe prior over the Laplace prior are primarily:

    \begin{vfilleditems}
        \small
        \item \textbf{shrinkage}:
            The Horseshoe prior has infinitely heavy tails and an infinite spike at zero.
            Parameters estimated under the Horseshoe prior can be shrunken towards zero
            more aggressively than under the Laplace prior,
            promoting sparsity without sacrificing the ability to detect true non-zero signals.
        \item \textbf{signal} detection:
            Due to its heavy tails,
            the Horseshoe prior does not overly penalize large values,
            which allows significant effects to stand out even in the presence
            of many small or zero effects.
        \item \textbf{uncertainty} quantification:
            With its heavy-tailed nature,
            the Horseshoe prior better captures uncertainty in parameter estimates,
            especially when the truth is close to zero.
        \item \textbf{regularization}: In high-dimensional settings where the number of
            predictors can exceed the number of observations,
            the Horseshoe prior acts as a strong regularizer,
            automatically adapting to the underlying sparsity level without the
            need for external tuning parameters.
    \end{vfilleditems}
\end{frame}

\begin{frame}{Effective Shinkrage Comparison}
    Makes more sense to compare the shinkrage effects of the proposed approaches
    so far.
    Assume for now that $\sigma^2 = \tau^2 = 1$,
    and define $\kappa_i = \frac{1}{1 + \lambda_i^2}$.
    Then $\kappa_i$ is a random shrinkage coefficient,
    and can be interpreted as the amount of weight that the posterior mean for
    $\beta_i$ places on $0$ once the data $\textbf{y}$ have been observed:

    $$
        \operatorname{E}(\beta_i \mid y_i, \lambda_i^2) = 
        \left( \frac{\lambda_i^2}{1 + \lambda_i^2} \right) y_i +
        \left( \frac{1}{1 + \lambda_i^2} \right) 0 =
        (1 - \kappa_i) y_i
    $$
\end{frame}

\begin{frame}{Effective Shinkrage Comparison\footnote{spike-and-slab with $p=\frac{1}{2}$
              would be very similar to Horseshoe but with discontinuities.}}
	\centering
	\begin{tikzpicture}
		\begin{axis}[every axis plot, line width=2pt,
				ylabel=PDF,
                xlabel=$\kappa$,
				domain=0.001:0.999,samples=200,
				axis x line*=bottom, % no box around the plot, only x and y axis
				axis y line*=left, % the * suppresses the arrow tips
				enlarge x limits=true, % extend the axes a bit
			]

			\addplot [blue] {beta(0.5,0.5)};
			\addlegendentry{Laplace}
			\addplot [red] {shrinkagelaplace(0.5)};
			\addlegendentry{Horseshoe}
		\end{axis}
	\end{tikzpicture}
\end{frame}

\subsubsection{Horseshoe+ Prior}
\begin{frame}{Shinkrage Priors -- Horseshoe+}
    Natural extension from the Horseshoe that has improved performance with
    highly sparse data \parencite{bhadra2015horseshoe}.

    \vfill

    Just introduce a new half-Cauchy mixing variable $\eta_i$ in the Horseshoe:

    $$
        \begin{aligned}
            \beta_i \mid \lambda_i, \eta_i, \tau &\sim \text{Normal}(0, \lambda_i) \\
            \lambda_i \mid \eta_i, \tau &\sim \text{Cauchy}^+ (0, \tau \eta_i) \\
            \eta_i &\sim \text{Cauchy}^+ (0, 1)
        \end{aligned}
    $$

    where:
    \begin{vfilleditems}
        \item $\tau$: \textit{global} shrinkage parameter
        \item $\lambda_i$: \textit{local} shrinkage parameter
        \item $\eta_i$: additional \textit{local} shrinkage parameter
        \item $\text{Cauchy}^+$ is the half-Cauchy distribution for the
            standard deviation $\lambda_i$ and $\eta_i$
    \end{vfilleditems}
\end{frame}

\subsubsection{Regularized Horseshoe Prior}
\begin{frame}{Shinkrage Priors -- Regularized Horseshoe}
    The Horseshoe and Horseshoe+ guarantees that the strong signals will not
    be overshrunk.
    However, this property can also be harmful,
    especially when the parameters are weakly identified.

    \vfill

    The solution, Regularized Horseshoe \parencite{piironen2017horseshoe}
    (also known as the ``Finnish Horseshoe''),
    is able to control the amount of shrinkage for the largest coefficient.
\end{frame}

\begin{frame}{Shinkrage Priors -- Regularized Horseshoe}
    $$
        \begin{aligned}
            \beta_i \mid \lambda_i, \tau, c &\sim \text{Normal} \left( 0, \sqrt{\tau^2 \tilde{\lambda_i}^2} \right) \\
            \tilde{\lambda_i}^2 &= \frac{c^2 \lambda_i^2}{c^2 + \tau^2 \lambda_i^2} \\
            \lambda_i &\sim \text{Cauchy}^+ (0, 1)
        \end{aligned}
    $$

    where:
    \begin{vfilleditems}
        \small
        \item $\tau$: \textit{global} shrinkage parameter
        \item $\lambda_i$: \textit{local} shrinkage parameter
        \item $c > 0$: regularization constant
        \item $\text{Cauchy}^+$ is the half-Cauchy distribution for the
            standard deviation $\lambda_i$
    \end{vfilleditems}
    \small
    Note that when $\tau^2 \lambda_i^2 \ll c^2$ (coefficient $\beta_i \approx 0$),
    then $\tilde{\lambda_i}^2 \to \lambda_i^2$;
    and when $\tau^2 \lambda_i^2 \gg c^2$ (coefficient $\beta_i$ far from $0$),
    then $\tilde{\lambda_i}^2 \to \frac{c^2}{\tau^2}$ and $\beta_i$ prior
    approaches $\text{Normal}(0,c)$.
\end{frame}

\subsubsection{R2-D2 Prior}
\begin{frame}{Shinkrage Priors -- R2-D2}
    Still, we can do better.
    The \textbf{R2-D2}\footnote{
        $R^2$-induced Dirichlet Decomposition
    } prior \parencite{zhang2022bayesian} has heavier tails and
    higher concentration around zero than the previous approaches.

    The idea is to, instead of specifying a prior on $\boldsymbol{\beta}$,
    we construct a prior on the coefficient of determination $R^2$\footnote{
        square of the correlation coefficient between the dependent variable
        and its modeled expectation.}.
    Then using that prior to ``distribute'' throughout the $\boldsymbol{\beta}$. 
\end{frame}

\begin{frame}{Shinkrage Priors -- R2-D2}
    $$
    \begin{aligned}
        R^2 &\sim \text{Beta} \left( \mu_{R^2} \sigma_{R^2}, (1 - \mu_{R^2}) \sigma_{R^2} \right) \\
        \boldsymbol{\phi} &\sim \text{Dirichlet}(J, 1) \\
        \tau^2 &= \frac{R^2}{1 - R^2} \\
        \boldsymbol{\beta} &= Z \cdot \sqrt{\boldsymbol{\phi} \tau^2}
    \end{aligned}
    $$
    where:
    \begin{vfilleditems}
        \small
        \item $\tau$: \textit{global} shrinkage parameter
        \item $\boldsymbol{\phi}$: proportion of total variance allocated to each covariate,
            can be interpreted as the \textit{local} shrinkage parameter
        \item $\mu_{R^2}$ is the mean of the $R^2$ parameter, generally $\frac{1}{2}$
        \item $\sigma_{R^2}$ is the precision of the $R^2$ parameter, generally $2$
        \item $Z$ is the standard Gaussian, i.e. $\text{Normal}(0, 1)$
    \end{vfilleditems}
\end{frame}
